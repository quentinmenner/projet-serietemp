% Options for packages loaded elsewhere
\PassOptionsToPackage{unicode}{hyperref}
\PassOptionsToPackage{hyphens}{url}
%
\documentclass[
  landscape]{article}
\usepackage{amsmath,amssymb}
\usepackage{lmodern}
\usepackage{iftex}
\ifPDFTeX
  \usepackage[T1]{fontenc}
  \usepackage[utf8]{inputenc}
  \usepackage{textcomp} % provide euro and other symbols
\else % if luatex or xetex
  \usepackage{unicode-math}
  \defaultfontfeatures{Scale=MatchLowercase}
  \defaultfontfeatures[\rmfamily]{Ligatures=TeX,Scale=1}
\fi
% Use upquote if available, for straight quotes in verbatim environments
\IfFileExists{upquote.sty}{\usepackage{upquote}}{}
\IfFileExists{microtype.sty}{% use microtype if available
  \usepackage[]{microtype}
  \UseMicrotypeSet[protrusion]{basicmath} % disable protrusion for tt fonts
}{}
\makeatletter
\@ifundefined{KOMAClassName}{% if non-KOMA class
  \IfFileExists{parskip.sty}{%
    \usepackage{parskip}
  }{% else
    \setlength{\parindent}{0pt}
    \setlength{\parskip}{6pt plus 2pt minus 1pt}}
}{% if KOMA class
  \KOMAoptions{parskip=half}}
\makeatother
\usepackage{xcolor}
\IfFileExists{xurl.sty}{\usepackage{xurl}}{} % add URL line breaks if available
\IfFileExists{bookmark.sty}{\usepackage{bookmark}}{\usepackage{hyperref}}
\hypersetup{
  pdftitle={Projet séries temporelles linéaires},
  pdfauthor={Antoine Joubrel et Quentin Menner},
  hidelinks,
  pdfcreator={LaTeX via pandoc}}
\urlstyle{same} % disable monospaced font for URLs
\usepackage[left=1.5cm,right=1.5cm,top=2cm,bottom=2cm]{geometry}
\usepackage{graphicx}
\makeatletter
\def\maxwidth{\ifdim\Gin@nat@width>\linewidth\linewidth\else\Gin@nat@width\fi}
\def\maxheight{\ifdim\Gin@nat@height>\textheight\textheight\else\Gin@nat@height\fi}
\makeatother
% Scale images if necessary, so that they will not overflow the page
% margins by default, and it is still possible to overwrite the defaults
% using explicit options in \includegraphics[width, height, ...]{}
\setkeys{Gin}{width=\maxwidth,height=\maxheight,keepaspectratio}
% Set default figure placement to htbp
\makeatletter
\def\fps@figure{htbp}
\makeatother
\setlength{\emergencystretch}{3em} % prevent overfull lines
\providecommand{\tightlist}{%
  \setlength{\itemsep}{0pt}\setlength{\parskip}{0pt}}
\setcounter{secnumdepth}{-\maxdimen} % remove section numbering
\ifLuaTeX
  \usepackage{selnolig}  % disable illegal ligatures
\fi

\title{Projet séries temporelles linéaires}
\author{Antoine Joubrel et Quentin Menner}
\date{}

\begin{document}
\maketitle

{
\setcounter{tocdepth}{2}
\tableofcontents
}
\hypertarget{partie-i-les-donnuxe9es}{%
\subsection{Partie I : Les données}\label{partie-i-les-donnuxe9es}}

\hypertarget{question-1}{%
\subsubsection{Question 1}\label{question-1}}

La série étudiée représente la production mensuelle de biens
manufacturés en France métropolitaine entre janvier 1985 et janvier
2000. C'est une série agrégée corrigée des variations saisonnières et
des jours ouvrés (CVS-CJO) contenant 181 observations. L'unité de mesure
est un indice de base 100 en 1990. La série est de la forme suivante :

\hypertarget{question-2}{%
\subsubsection{Question 2}\label{question-2}}

\begin{enumerate}
\def\labelenumi{\alph{enumi})}
\tightlist
\item
  Nous sommes d'abord demandé si la série avait une saisonnalité. Une
  façon de le voir est de regarder la répartition de l'indice pour
  chaque mois :
\end{enumerate}

\includegraphics{Rapport_projet_-Antoine_Joubrel_Quentin_Menner-_files/figure-latex/unnamed-chunk-1-1.pdf}

Nous avons conclu qu'aucune saisonnalité semble se dégager.

\begin{enumerate}
\def\labelenumi{\alph{enumi})}
\setcounter{enumi}{1}
\tightlist
\item
  Toutefois la série possède une tendance nette à la hausse malgré une
  baisse de l'indice entre 1990 et 1994.
\end{enumerate}

\includegraphics{Rapport_projet_-Antoine_Joubrel_Quentin_Menner-_files/figure-latex/unnamed-chunk-2-1.pdf}

Elle est nette quand on réalise une régression linéaire de la série sur
le temps (coefficient de 0.087).

Nous avons donc différencié la série puis réalisé des tests de
stationnarité sur cette nouvelle série.

Les test ADF et PP rejette l'hypothèse de racine unité et le test KPSS
ne rejette pas l'hypothèse de stationnarité de la série. Nous
considérons donc que la série différenciée est stationnaire.

\hypertarget{question-3}{%
\subsubsection{Question 3}\label{question-3}}

\includegraphics{Rapport_projet_-Antoine_Joubrel_Quentin_Menner-_files/figure-latex/unnamed-chunk-3-1.pdf}
\includegraphics{Rapport_projet_-Antoine_Joubrel_Quentin_Menner-_files/figure-latex/unnamed-chunk-3-2.pdf}

\hypertarget{partie-ii-moduxe8les-arma}{%
\subsection{Partie II : Modèles ARMA}\label{partie-ii-moduxe8les-arma}}

\hypertarget{question-4}{%
\subsubsection{Question 4}\label{question-4}}

Une caractéristique importante de notre série temporelle (non
stationnarisée) est l'autocorrélation des résidus. On observe en effet
qu'ils sont linéairement décroissants et en dehors de l'intervalle de
confiance comme c'est le cas dans un AR pure. Nous avons donc fait
l'hypothèse que la série est un AR pure.

\begin{verbatim}
## Registered S3 method overwritten by 'quantmod':
##   method            from
##   as.zoo.data.frame zoo
\end{verbatim}

\includegraphics{Rapport_projet_-Antoine_Joubrel_Quentin_Menner-_files/figure-latex/unnamed-chunk-4-1.pdf}

Continuons tout de même la méthode de Box-Jenkins pour voir si notre
hypothèse se vérifie.

Les graphiques de l'ACF et du PACF de la série stationnaire donnent
comme valeur maximale des coefficients AR et MA : 1.

\includegraphics{Rapport_projet_-Antoine_Joubrel_Quentin_Menner-_files/figure-latex/unnamed-chunk-5-1.pdf}

On évalue donc 4 modèles : ARMA(0,0), ARMA(1,0) (ou AR(1)), ARMA(0,1)
(ou MA(1)) et ARMA(1,1) en regardant d'abord l'autocorrélation des
résidus et la significativité des coefficients.

\begin{itemize}
\tightlist
\item
  Pour le modèle ARMA(0,0) (bruit blanc), on estime le coefficient de la
  constante:
\end{itemize}

\begin{verbatim}
## # A tibble: 1 x 5
##   term      estimate std.error statistic p.value
##   <chr>        <dbl>     <dbl>     <dbl>   <dbl>
## 1 intercept    0.137    0.0836      1.64   0.102
\end{verbatim}

Ce paramètre n'est pas significative.

Puis nous avons étudié l'autocorrélation des résidus par des tests de
Ljung-Box :

\begin{verbatim}
##       lag b.statistic b.p.value   
##  [1,] 1   NA          NA          
##  [2,] 2   29.05626    7.030669e-08
##  [3,] 3   30.76253    2.089307e-07
##  [4,] 4   32.15385    4.856948e-07
##  [5,] 5   32.31955    1.645929e-06
##  [6,] 6   38.24597    3.367374e-07
##  [7,] 7   40.44434    3.725091e-07
##  [8,] 8   48.98071    2.288786e-08
##  [9,] 9   55.46514    3.584941e-09
## [10,] 10  55.90873    8.170861e-09
\end{verbatim}

Les résidus ne sont pas du tout autocorrélés. Mais\textbf{on ne retient
pas ce modèle} car le coefficient n'est pas significatif.

\begin{itemize}
\tightlist
\item
  On procède de même pour le modèle AR(1) :
\end{itemize}

\begin{verbatim}
## # A tibble: 2 x 5
##   term      estimate std.error statistic      p.value
##   <chr>        <dbl>     <dbl>     <dbl>        <dbl>
## 1 ar1         -0.388    0.0694     -5.59 0.0000000222
## 2 intercept    0.134    0.0556      2.40 0.0163
\end{verbatim}

Les coefficients sont significatifs aux seuils usuels.

\begin{verbatim}
##       lag b.statistic b.p.value  
##  [1,] 1   NA          NA         
##  [2,] 2   NA          NA         
##  [3,] 3   4.091048    0.04311091 
##  [4,] 4   4.720079    0.09441648 
##  [5,] 5   5.985802    0.112303   
##  [6,] 6   13.86031    0.007754497
##  [7,] 7   14.06167    0.0152224  
##  [8,] 8   18.53717    0.005021067
##  [9,] 9   21.7933     0.002757393
## [10,] 10  21.91809    0.005069882
## [11,] 11  24.59587    0.0034523  
## [12,] 12  25.77409    0.004056201
## [13,] 13  27.38773    0.004014003
## [14,] 14  27.40036    0.006763869
## [15,] 15  27.70803    0.009937022
\end{verbatim}

Globalement, les résidus ne sont pas corrélés entre eux (à part pour
l'ordre 5). Donc \textbf{le modèle AR(1) est retenu.}

De même \textbf{le modèle MA(1) est retenu} car les résidus sont encore
moins autocorrélés et les coefficients sont significatifs. Toutefois
\textbf{le modèles ARMA(1,1) n'est pas retenu} car les coefficients ne
sont pas tous significatifs.

Pour départager le modèle AR(1) du modèle MA(1) nous avons comparé les
critères d'informations (AIC et BIC) qui montre que le modèle AR(1) est
meilleur.

Est-il valide ? Pour cela, nous avons observé la répartition des
résidus.
\includegraphics{Rapport_projet_-Antoine_Joubrel_Quentin_Menner-_files/figure-latex/unnamed-chunk-10-1.pdf}

\begin{verbatim}
## 
##  Ljung-Box test
## 
## data:  Residuals from ARIMA(1,0,0) with non-zero mean
## Q* = 44.817, df = 22, p-value = 0.002799
## 
## Model df: 2.   Total lags used: 24
\end{verbatim}

\hypertarget{question-5}{%
\subsubsection{Question 5}\label{question-5}}

Nous supposons que donc la production mensuelle de biens manufacturés
suit un \textbf{ARIMA(1,1,0)} car la série avait besoin d'être
différencié une fois pour être stationnaire. Le coefficients lié à
l'AR(1) est estimé à \textbf{-0.388} et la constante est de
\textbf{0.134}. La formule de la série est donc :

\[
X_t = 0.134 + 0.612X_{t-1} + 0.388X_{t-2} + \epsilon_t
\] avec \(\epsilon_t\) un bruit blanc gaussien.

\hypertarget{partie-iii-pruxe9vision}{%
\subsection{Partie III : Prévision}\label{partie-iii-pruxe9vision}}

\hypertarget{question-6}{%
\subsubsection{Question 6}\label{question-6}}

Comme nous avons réalisé une différentiation d'un mois à l'autre dans la
question 2, nous prenons ici le modèle AR(1) de la forme suivante : \[
\begin{align*}
\Delta X_{t+1} &= a_{0} \Delta X_{t} + \epsilon_{t} \\
X_{t+1} &= (a_{0}+1)X_{t} - a_{0}X_{t-1} + \epsilon_{t}
\end{align*}
\] Avec \((\epsilon_{t})_{t} \sim {\sf Norm}(0,1)\). Avec une
probabilité de \(\alpha\%\),
\(q_{\frac{\alpha}{2}} \leq \mid\epsilon_{t}\mid \leq q_{1-\frac{\alpha}{2}}\).
Ce qui donne : \$\$ \begin{align*}
q_{\frac{\alpha}{2}} + (a_{0}+1)X_{T} - a_{0}X_{T-1} \leq &X_{T+1} \leq q_{1-\frac{\alpha}{2}} + (a_{0}+1)X_{T} - a_{0}X_{T-1} \\ \\

\text{Et : } q_{\frac{\alpha}{2}} + (a_{0}+1)X_{T+1} - a_{0}X_{T} \leq &X_{T+2} \leq q_{1-\frac{\alpha}{2}} + (a_{0}+1)X_{T+1} - a_{0}X_{T} \\

q_{\frac{\alpha}{2}} + (a_{0}+1)(q_{\frac{\alpha}{2}} + (a_{0}+1)X_{T} - a_{0}X_{T-1}) - a_{0}X_{T} \leq &X_{T+2} \leq q_{1-\frac{\alpha}{2}} + (a_{0}+1)(q_{1-\frac{\alpha}{2}} + (a_{0}+1)X_{T} - a_{0}X_{T-1}) - a_{0}X_{T} \\

(a_{0}+2)q_{\frac{\alpha}{2}} + (a_{0}^2+a_{0}+1)X_{T} - a_{0}(a_{0}+1)X_{T-1} \leq &X_{T+2} \leq (a_{0}+2)q_{1-\frac{\alpha}{2}} + (a_{0}^2+a_{0}+1)X_{T} - a_{0}(a_{0}+1)X_{T-1} \\

\end{align*} \$\$ D'après la partie II, \(a_{0} = -0,388\). Donc on a
comme région de confiance de niveau \(\alpha\):

\(X_{T+1} \in \left[q_{\frac{\alpha}{2}} + 0.612X_{T} + 0.388X_{T-1}, q_{1-\frac{\alpha}{2}} + 0.612X_{T} + 0.388X_{T-1}\right]\)

Et :
\(X_{T+2} \in \left[q_{\frac{\alpha}{2}} + 0.612X_{T+1} + 0.388X_{T}, q_{1-\frac{\alpha}{2}} + 0.612X_{T+1} + 0.388X_{T}\right]\)

Ce qui donne comme intervalle de confiance pour \(X_{t+2}\):
\(X_{T+2} \in \left[1.612q_{\frac{\alpha}{2}} + 0.762544X_{T} + 0.237456X_{T-1}, 1.612q_{1-\frac{\alpha}{2}} + 0.762544X_{T} + 0.237456X_{T-1}\right]\)

\hypertarget{question-7}{%
\subsubsection{Question 7}\label{question-7}}

Pour obtenir cette région de confiance, nous utilisons les hypothèses
que les résidus suivent une loi normale centrée réduite, que la série
temporelle est considérée comme stationnaire une fois la différenciation
effectuée et qu'elle peut être modélisée par un modèle ARIMA de
paramètres (1,1,0).

\hypertarget{question-8}{%
\subsubsection{Question 8}\label{question-8}}

On prend ici une région de confiance de niveau 95\%. Donc
\(q_{2.5}=-1.96\) et \(q_{97.5}=1.96\). On considère T et T-1 comme
étant respectivement la dernière et l'avant-dernière période de notre
série temporelle ce qui donne \(X_{T}=112.3\) et \(X_{T-1}=112.6\). Donc
\(X_{T+1} \in \left[110.4564,114.3764\right]\) et
\(X_{T+2} \in \left[41.6124 - 0.612X_{T+1},45.5324 - 0.612X_{T+1}\right]\)

Cela nous donne la région de confiance suivante pour \(X_{T+2}\) en
fonction de \(X_{T+1}\) :
\includegraphics{Rapport_projet_-Antoine_Joubrel_Quentin_Menner-_files/figure-latex/unnamed-chunk-11-1.pdf}

\hypertarget{question-9}{%
\subsubsection{Question 9}\label{question-9}}

On cherche à déterminer une méthode permettant de prédire à \(X_{T+1}\)
à partir d'une autre série \((Y_{t})_{1 \leq t \leq T+1}\) et plus
précisément de sa valeur à la T+1ème période \(Y_{T+1}\).

On sait que \((Y_{t})_{1 \leq t \leq T+1}\) est une série stationnaire.
De même dans notre cas, \((\Delta X_{t})_{t}\) est aussi stationnaire.
Comme on considère que l'on a \(Y_{T+1}\) avant \(X_{T+1}\), on peut
donc réaliser la régression suivante : \begin{align*}
\Delta X_{T+1} &= \beta Y_{T+1} + c + u_{t} \\
X_{T+1} &= \beta Y_{T+1} + X_{T} + c + u_{t} \\
\text{Avec : }\beta &= \frac{Cov(X_{t},Y_{t})}{Var(X_{t})} \\
\text{et }c &= EY_{t}-\beta EX_{t} \\
\end{align*} Pour pouvoir réaliser cette régression il faut s'assurer
que la condition suivante est respectée : \begin{align*}
\beta \neq 0 &\iff Cov(X_{t},Y_{t}) \neq 0 \\
\end{align*} On doit en plus considérer comme condition que les
observations sont comme un échantillon donc que l'écart type estimé de
\(\hat{\beta}_{n}\) notée \(se_{n}\) est juste ce qui ne serait pas le
cas normalement. Dans ce cas, on peut réaliser un test de
\(H_{0} : \beta = 0\) contre \(H_{1} : \beta \neq 0\). Pour tester
\(H_{0}\) au niveau \(\alpha\), nous considérons la t-statistique
\(t_{n}=\frac{\hat{\beta}_{n}}{se_{n}}\) et la région critique :
\(W_{\alpha}=\{\mid t_{j} \mid > q_{1-\frac{\alpha}{2}}\},\) où
\(q_{1-\frac{\alpha}{2}}\) est le quantile d'ordre
\(1-\frac{\alpha}{2}\) de \(N(0,1)\).

\end{document}
